% !TEX encoding = UTF-8 Unicode

\documentclass[a4paper]{article}

\usepackage{color}
\usepackage{url}
\usepackage[T2A]{fontenc} % enable Cyrillic fonts
\usepackage[utf8]{inputenc} % make weird characters work
\usepackage{graphicx}

\usepackage[english,serbian]{babel}
%\usepackage[english,serbianc]{babel} %ukljuciti babel sa ovim opcijama, umesto gornjim, ukoliko se koristi cirilica

\usepackage[unicode]{hyperref}
\hypersetup{colorlinks,citecolor=green,filecolor=green,linkcolor=blue,urlcolor=blue}

%\newtheorem{primer}{Пример}[section] %ćirilični primer
\newtheorem{primer}{Primer}[section]

\begin{document}

\title{Temporalne logike\\ \small{Seminarski rad u okviru kursa\\Metodologija stručnog i naučnog rada\\ Matematički fakultet}}

\author{Djordje Todorovic, Stefan Zaric, Maksim Djurdjevac\\ mi12090@matf.bg.ac.rs, drugog (trećeg) autora}
\date{8.~april 2017.}
\maketitle

\abstract{
Ovaj seminarski rad obradjuje temu temporalne logike, zapravo upotrebu logike u racunarstvu.
Preciznije su opisane osnove linarne temporalne logike (LTL) i logike stabla 
izracunavanja (CTL), osnovne primene ovih logika, takodje i najpoznatiji alati za proveravanje modela,
kao i ukratko mogucnosti alata ESBMC i neki primeri njegovih upotreba.

\tableofcontents

\newpage

\section{Uvod}
\label{sec:uvod}

Glavni cilj upotrebe logike u racunarstvu je da se razviju jezici koji modeluju situacije koje srecemo kao profesionalci
u toj oblasti, u nekom formalnom obliku. Razlozi za to su da damo formalne pravila za prepoznavanje koji su argumenti ispravni,tj. validni, a koji ne. uska je povezanost pojmova Temporalna logika i iskazna logika, gde takodje imamo jasno definisanu sintaksu i semantiku. Jezici u matematici su pretezno simbolicki, sto ce reci da kombinovanjem istih dobijamo izraze koji predstavljaju odredjene matematicke objekte. Sintaksa jezika definise skup validnih simbola koji se mogu upotrebljavati kako bi se kreirali takodje validni izrazi. Semtantika se bavi tumacenjem izraza. Iskazna logika (i predikarska logika) je dobra za opisivanje matematickih problema koje srecemo u teoriji, ali cesto imamo potrebu za opisivanjem i modelovanjem problema koje srecemo svakodnevno u zivotu. Temporalne logike su logike koje opisuju rezonovanja u kojima je ukljuceno i vreme. Primena ovih logika je sve veca kako kod softverskih resenja, tako i kod kreiranja samog hardvera.

\newpage
\section{Linearne temporalne logike}
\label{sec:LTL}

\newline
Kao sto smo vec napomenuli, Temporalne logike ukljucuju u razmatranje i vreme, pa stoga cemo prvo opisati LTL model, linearne temporalne logike. Linearne temporalne logike predstavljaju beskonacan niz sekvenci gde u svakom trenutku svaka od njih ima svog naslednika, na osnovu linearne vremenske perspektive. Uvodimo i pojam 'linearno temporalno svojstvo' koje predstavlja temporalnu logicku formulu koja opisuje skup beskonacnih sekvenci koje su tacne, tj. imaju vrednost 'True'. 

Svrha LTL-a jeste da se svojstva napisana u prirodnom jeziku prevedu u LTL uz pomoc precizno definisane sintakse. Ovu logiku odlikuje pojava razlicitih operatora, cijom upotrebom mozemo opisati razne modele vremena, ali nama ce za sada biti interesantni samo operatori koji opisuju linearne sekvence. Postoje i tkzv. univerzalni logicki veznici, kojima mozemo opisati svako temporalnu logiku.\newline
\newline
Sintaksa
\newline
\newline
LTL formula je sacinjena od \newline
	1) konacnog skupa atoma (a, b, x, y, ....) \newline
	2) osnovnih logickih operatora (¬ (negation) , ∧ (conjunction)) \newline
	3) osnovnih temporalnih operatora (O (next) , U (until) , true) \newline
	4) Postoje i dodatni logicki operatori (∨ (disjunction), →(implication), ↔(equivalence)) \newline
	5) Postoje i dodatni temporalni operatori ( □ (”always”) ♦ (”eventually”)) \newline

Kombinacijom navedenih mozemo dobiti slozenije operatore.

Navescemo jos neka pravila: \newline

	*) ♦ -> "F" Konacno sto znaci nesto u buducnosti \newline
	*) □ -> "G" Globalno sto znaci globalno u buducnosti \newline
	*) ○ -> "X" Sledeci put \newline

LTL moze biti prosireno sa operatorima proslosti \newline
	
	*) □-1 -> Uvek u proslosti \newline
	*) ♦-1 -> Ponekad u proslosti \newline
	*) ○-1 -> Prethodna sekvenca \newline

*) Weak until (a W b) \newline
	-a ostaje True dok b ne postane True, pri tom moze zauvek ostati True (ne zahteva se da b ikad postane True) \newline
*) Release (a R b) \newline
	-b ostaje True dok a ne postane True, pri tom moze zauvek ostati True (ne zahteva se da b ikad postane True) znaci suprotno od prethodnog \newline

Semantika
\newline
\newline
Sintaksa logike je definisala kako nesto mozemo zapisati ispravno, a semantika daje znacenje istom.
Vrste sistema koje obicno modelujemo uz pomoc LTL-a mozemo nazvati translacionim sistemima (TS). Razaznajemo pojmove stanica, sto predstavlja staticku strukturu i trannziciju, sto predstavlja dinamicku strukturu.
LTL formula φ oznacava svojstva puteva (traces) i jedan konkretan put moze biti potpuna LTL formula a i ne mora.
Semantika formule φ je definisana kao jezik(φ) gde on sadrzi beskonacnan skup reci koje zadovoljavaju φ.

TS zadovoljava LT svojstvo P ako svi njegove stanice zadovoljavaju P, tj. ako sva njegova ponasanja jesu prihvatljiva. A stanica zadovoljava P kadgod svi putevi koji su krenuli iz te stanice zadovoljavaju P.

TS zadovoljava ϕ ako TS zadovoljava LT svojstva dedfinisana na Jezik(ϕ).
}

\section{Logike stabla izracunavanja}
\label{sec:CTL}

CTL (Logike stabla)
\newline
\newline
U LTL analizi smo zapazili da je LTL formula evaluirana preko puteva(eng. paths). Definisali smo da je stanica(state) sistema
zadovoljava jednu LTL formulu ako svi putevi iz te stanice(state) zadovoljavaju istu. U ovoj strukturi drveta buduci koraci jos nisu
odredjeni, jer postoji vise puteva koji potencijalno mogu biti 'iskoriscena'.
CTL nam omogucava da prebrodimo neke nedostatke LTL metoda. Kao i u LTL imamo operatore U, F, G i X, takodje imamo i kvantifikatore
A i E koji opisuju sve puteve(all paths) i postojece puteve (exists paths). 
\newline
\newline
Sintaksa
\newline
\newline
CTL (eng Computation tree logic) koristi logiku stabla pri izracunavanju. 
Kao pre, radimo sa konacnim skupom atomickih formula (kao sto su p, q, r,...,p1,p2,p3...).

Definisemo CTL formulu:

ϕ ::= T | neT | p | (neϕ) | (ϕ /\ ϕ) | (ϕ \/ ϕ) | (ϕ -> ϕ) | AXϕ | EXϕ | Afϕ | EFϕ | AGϕ | EGϕ | A[ϕ U ϕ] | E[ϕ U ϕ]

gde je p iz konackog skupa atomickih formula.
\newline
\newline
Semantika
\newline
\newline
CTL formula je interpretirana preko tranzicionog sistema (TS). Neka je M = (S, ->, L) neki model, s e S i ϕ CTL formula.
Definicija modela bi bila sledeca

	1) Ako je ϕ atomicka, zadovoljivost je razresena preko L \newline
	2) Ako veznik najviseg prioriteta formule ϕ je bulovski veznik tada se pitanje zadovoljivosti razresava
	   preko uobicajne istinitosne tablice i rekurzije nad drvetom formule ϕ \newline
	3) Ako je veznik najviseg prioriteta neki operator koji pocinje sa A, tada zadovoljivost jeste ispunjena ako
	   svi putevi iz s zadovoljavaju LTL formulu koja se dobija kada izbacimo simbol A \newline
	4) Slicno ako je veznik najviseg prioriteta neki operator koji pocinje sa E, ada zadovoljivost jeste ispunjena ako
	   svi putevi iz s zadovoljavaju LTL formulu koja se dobija kada izbacimo simbol E \newline

\newpage
\begin{primer}
Problem zaustavljanja (eng.~{\em halting problem}) je neodlučiv \cite{haltingproblem}.
\end{primer}

\begin{primer}
Za prevođenje programa napisanih u programskom jeziku C može se koristiti GCC kompajler \cite{gcc}.
\end{primer}

\begin{primer}
 Da bi se ispitivala ispravost softvera, najpre je potrebno precizno definisati njegovo ponašanje \cite{laski2009software}. 
\end{primer}

Reference koje se koriste u ovom tekstu zadate su u datoteci {\em seminarski.bib}. Prevođenje u pdf format u Linux okruženju može se uraditi na sledeći način:
\begin{verbatim}
pdflatex TemaImePrezime.tex 
bibtex TemaImePrezime.aux 
pdflatex TemaImePrezime.tex 
pdflatex TemaImePrezime.tex 
\end{verbatim}
Prvo latexovanje je neophodno da bi se generisao {\em .aux} fajl. {\em bibtex} proizvodi odgovarajući {\em .bbl} fajl koji se koristi za generisanje literature. 
Potrebna su dva prolaza (dva puta pdflatex) da bi se reference ubacile u tekst (tj da ne bi ostali znakovi pitanja umesto referenci). Dodavanjem novih referenci potrebno je ponoviti ceo postupak.  


Broj naslova i podnaslova je proizvoljan. Neophodni su samo Uvod i Zaključak. Na poglavlja unutar teksta referisati se po potrebi. 
\begin{primer}
U odeljku \ref{sec:naslov1} precizirani su osnovni pojmovi, dok su zaključci dati u odeljku \ref{sec:zakljucak}.
\end{primer}

Još jednom da napomenem da nema razloga da pišete:
\begin{verbatim}
\v{s} i \v{c} i \'c ...
\end{verbatim}
Možete koristiti srpska slova
\begin{verbatim}
š i č i ć ... 
\end{verbatim}


Ovde pišem uvodni tekst.
Ovde pišem uvodni tekst. 
Ovde pišem uvodni tekst. 
Ovde pišem uvodni tekst. 


\section{Slike i tabele}
\label{slike_i_tabele}

Slike i tabele treba da budu u svom okruženju, sa odgovarajućim naslovima, obeležene labelom da koje omogućava referenciranje. 

\begin{primer} Ovako se ubacuje slika. Obratiti pažnju da je dodato i 
\begin{verbatim}
\usepackage{graphicx}
\end{verbatim}

\begin{figure}[h!]
\begin{center}
\includegraphics[scale=0.75]{panda.jpg}
\end{center}
\caption{Pande}
\label{fig:pande}
\end{figure}

Na svaku sliku neophodno je referisati se negde u tekstu. Na primer, na slici \ref{fig:pande} prikazane su pande. 
\end{primer}

\begin{primer} I tabele treba da budu u svom okruženju, i na njih je neophodno referisati se u tekstu. Na primer, u tabeli \ref{tab:tabela1} su prikazana različita poravnanja u tabelama.

\begin{table}[h!]
\begin{center}
\caption{Razlčita poravnanja u okviru iste tabele ne treba koristiti jer su nepregledna.}
\begin{tabular}{|c|l|r|} \hline
centralno poravnanje& levo poravnanje& desno poravnanje\\ \hline
a &b&c\\ \hline
d &e&f\\ \hline
\end{tabular}
\label{tab:tabela1}
\end{center}
\end{table}

\end{primer}





\section{Prvi naslov}
\label{sec:naslov1}


Ovde pišem tekst. 
Ovde pišem tekst. 
Ovde pišem tekst. 
Ovde pišem tekst. 
Ovde pišem tekst. 
Ovde pišem tekst. 
Ovde pišem tekst. 
Ovde pišem tekst. 


\subsection{Prvi podnaslov}
\label{subsec:podnaslov1}

Ovde pišem tekst. 
Ovde pišem tekst. 
Ovde pišem tekst. 
Ovde pišem tekst. 
Ovde pišem tekst. 
Ovde pišem tekst. 
Ovde pišem tekst. 

\subsection{Drugi podnaslov}
\label{subsec:podnaslov2}

Ovde pišem tekst. 
Ovde pišem tekst. 
Ovde pišem tekst. 
Ovde pišem tekst. 
Ovde pišem tekst. 
Ovde pišem tekst. 

\section{Drugi naslov}
\label{sec:naslov2}

Ovde pišem tekst. 
Ovde pišem tekst. 
Ovde pišem tekst. 
Ovde pišem tekst. 

\subsection{... podnaslov}
\label{subsec:podnaslovN}

Ovde pišem tekst. 
Ovde pišem tekst. 
Ovde pišem tekst. 
Ovde pišem tekst. 
Ovde pišem tekst. 
Ovde pišem tekst. 

\section{n-ti naslov}
\label{sec:naslovN}

Ovde pišem tekst. 
Ovde pišem tekst. 
Ovde pišem tekst. 
Ovde pišem tekst. 
Ovde pišem tekst. 

\subsection{... podnaslov}
\label{subsec:podnaslovK}

Ovde pišem tekst. 
Ovde pišem tekst. 
Ovde pišem tekst. 
Ovde pišem tekst. 
Ovde pišem tekst. 

\subsection{... podnaslov}
\label{subsec:podnaslovM}

Ovde pišem tekst. 
Ovde pišem tekst. 
Ovde pišem tekst. 
Ovde pišem tekst. 
Ovde pišem tekst. 

\section{Poslednji naslov}
\label{sec:naslovM}

Ovde pišem tekst. 
Ovde pišem tekst. 
Ovde pišem tekst. 
Ovde pišem tekst. 
Ovde pišem tekst. 
Ovde pišem tekst. 
Ovde pišem tekst. 
Ovde pišem tekst. 
Ovde pišem tekst. 

\section{Zaključak}
\label{sec:zakljucak}

Ovde pišem zaključak. 
Ovde pišem zaključak. 
Ovde pišem zaključak. 
Ovde pišem zaključak. 
Ovde pišem zaključak. 
Ovde pišem zaključak. 
Ovde pišem zaključak. 
Ovde pišem zaključak. 
Ovde pišem zaključak. 
Ovde pišem zaključak. 
Ovde pišem zaključak. 
Ovde pišem zaključak. 


\addcontentsline{toc}{section}{Literatura}
\appendix
\bibliography{seminarski} 
\bibliographystyle{plain}

\appendix
\section{Dodatak}
Ovde pišem dodatne stvari, ukoliko za time ima potrebe.
Ovde pišem dodatne stvari, ukoliko za time ima potrebe.
Ovde pišem dodatne stvari, ukoliko za time ima potrebe.
Ovde pišem dodatne stvari, ukoliko za time ima potrebe.
Ovde pišem dodatne stvari, ukoliko za time ima potrebe.


\end{document}
